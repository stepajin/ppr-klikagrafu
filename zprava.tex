\documentclass[12pt]{article}
\usepackage{epsf,epic,eepic,eepicemu}
\usepackage[utf8]{inputenc}

\usepackage[czech]{babel}
\usepackage{wrapfig}
\usepackage{graphicx}
\usepackage{epstopdf}
\usepackage{multirow}
\usepackage{listings}


% hack to fix cline
\usepackage{booktabs}

\usepackage{etoolbox}
\preto\tabular{\shorthandoff{-}}
% end of hack

\begin{document}

\begin{center}
  \bf Semestralní projekt MI-PAR 2014/2015:\\[5mm]
      Paralelní algoritmus pro řešení problému\\[5mm]
         Jindřich Štěpánek\\
         Václav Mach\\[2mm]
  magisterské studijum, FIT ČVUT, Kolejní 550/2, 160 00 Praha 6\\[2mm]
  \today
\end{center}

\section{Definice problému a popis sekvenčního algoritmu}

\subsection{Definice problému}

Zadaným problémem bylo nalezení maximální kliky na zadaném grafu.
Pojmem klika grafu je v teorii grafů označen takový maximální podgraf nějakého grafu, 
který je úplným grafem, tzn. jehož všechny vrcholy jsou spojeny hranou se všemi zbylými.
Velikost kliky je dána počtem vrcholů.

\begin{figure}[ht]
  \begin{center}
    \includegraphics[scale=0.5]{clique.pdf}
  \end{center}
  \caption{Největší klika (1,2,5) tohoto grafu je označena červeně.}
\end{figure}

Nalezení maximální kliky grafu tedy znamená, že je nutné postupně procházet celý stavový prostor uzlů grafu
a zkoumat, zda konkrétní množina uzlů tvoří úplný graf. Problém je možné řešit dvěma způsoby, 
buď postupným přidáváním nebo odebíráním uzlů. Rozhodli jsme se pro řešení pomocí přidávání.

\subsection{Vstup a výstup programu}

Vstup programu:
\begin{itemize}
  \item{$G(V,E)$ -- jednoduchý souvislý neorientovaný neohodnocený graf o $n$ uzlech a $m$ hranách}
  \item{$n$ -- přirozené číslo představující počet uzlů grafu $G$, $n >= 5$}
  \item{$k$ -- přirozené číslo řádu jednotek představující průměrný stupeň uzlu grafu $G$, $n >= k >= 3$}
  \item{$r$ -- kladné reálné číslo, $0 < r < 1$}    % TODO
\end{itemize}

Pro generování grafů jsme použili generátor s volbou typu grafu ''-t AD'', 
který vygeneruje souvislý neorientovaný neohodnocený graf. příklad vstupu pro parametry $n = 10, k = 4$: 

%\begin{lstlisting}
\begin{verbatim}
10
0010010000
0000001010
1000111111
0000000110
0010011111
1010100100
0110100010
0011110010
0111101100
0010100000
\end{verbatim}
%\end{lstlisting}

na prvním řádku vstupního souboru je uveden počet vrcholů, na dalších řádcích následuje samotný popis grafu v binární podobě.

Úkolem programu mělo být zjištění, zda graf $G$ obsahuje kliku o velikosti alespoň (rovné nebo větší) $r * n$ a nalézt největší takovou kliku. 
Výstupem programu měl být seznam uzlů tvořící kliku, popřípadě konstatování, že klika neexistuje. 


\subsection{Popis sekvenčího algoritmu}

Sekvenční algoritmus pro řešení tohoto problému je typu BB-DFS (branch-and-bound Depth-first search) 
s hloubkou stavového stromu omezenou na n. 
Cena řešení, která se maximalizuje, je velikost kliky vzhledem k zadané podmínce. 
Horní mez ceny řešení není známa. Algoritmus skončí, až prohledá celý stavový prostor.
Dolní mez je 2, pokud graf obsahuje aspoň 1 hranu.
Horní mez není známá, ale dá se odhadnout takto: Pokud $G$ obsahuje kliku o velikosti $x$, pak musí obsahovat $x$ vrcholů se stupněm větším nebo rovným $x-1$.

\subsection{Implementace}

Pro naši implementaci sekvenčního algoritmu jsme zvolili jazyk C++.
Program nejprve přečtě vstupní data ze souboru, který je prvním argumentem programu a následně inicializuje zásobník 
realizovaný pomocí třídy vector s datovým typem int.
Dále program už pouze prohledává stavový prostor, dokud má na zásobníku data ke zpracování
a testuje zda konkrétní daná množina tvoří úplný graf. 
Pokud algoritmus najde množinu uzlů, která tvoří kliku grafu a následně přidá uzel, se kterým již množina kliku netvoří, dále již v prohledávání této větvě nepokračuje.
Nejvetší nalezená kliku je v průběhu výpočtu uložena a po skončení prohledávání stavového prostoru je porovnána se zadanou podmínkou.

Z tabulky je zřejmé, že jsme se odchýlili od zadání ve vstupních datech parametrem $k$, který měl být podle zadání v řádu jednotek.
Tato odchylka byla z našeho pohledu nutná, pokud bychom měli dodržet zadání přesně, bylo by nutné hledat instance problému
v daném časovém intervalu témeř výhradně na základě parametru $n$. Protože toto se ukázalo jako velmi problematické při zachovaní 
parametru $k$ v řádu jednotek, rozhodli jsme se použít vzájemně ''vyvážené'' hodnoty obou parametrů.


%Popište problém, který váš program řeší. Jako výchozí použijte text
%zadání, který rozšiřte o přesné vymezení všech odchylek, které jste
%vůči zadání během implementace provedli (např.  úpravy heuristické
%funkce, organizace zásobníku, apod.). Zmiňte i případně i takové
%prvky algoritmu, které v zadání nebyly specifikovány, ale které se
%ukázaly jako důležité.  Dále popište vstupy a výstupy algoritmu
%(formát vstupních a výstupních dat). Uveďte tabulku nameřených časů
%sekvenčního algoritmu pro různě velká data.

\section{Popis paralelního algoritmu a jeho implementace v MPI}

\subsection{Popis paralelního algoritmu}

Paralelní algoritmus pro řešení tohoto problému je typu PBB-DFS-V. 
Algoritmus PBB-DFS-V lze popsat následovně: 
Všechny procesory vědí nebo se dozvědí hodnotu horní meze ceny řešení a hodnota dolní meze není známa. 
Pak stačí, aby si každý procesor lokálně udržoval informaci o svém dosud nejlepším řešení. 
Po vyprázdnění všech zásobníků se provede distribuované ukončení výpočtu pomocí modifikovaného ADUV algoritmu a 
pak pomocí paralelní redukce se ze všech nejlepších lokálních řešení vybere globálně nejlepší. 

\subsection{Implementace}

Pro dělení zásobníku jsme zvolili algoritmus D-ADZ, který je podle nás vhodný z důvodu
rozložení stavového prostoru. Stavy na dně zásobníku obsahují mnohem vetší podprostory, než stavy
na vrcholu zásobníku, proto tímto dělením dosáhneme efektivního rozložení práce mezi procesory.

V naší implementaci jsme se od výše popsaného algoritmu odchýlili v části dokončení výpočtu,
kdy by měly výpočetní uzly provést paralelní redukci pro získání globálního výsledku.
My jsme provedli úpravu, kdy procesory pošlou své nalezené výsledky mateřskému procesoru, který výsledky
následně zpracuje. Tato odchylka byla nutná, protože paralelní redukci lze použít pouze pro číselné výsledky 
a ne strukturovaná data jako v našem případě.

Jako algoritmus dárce jsme zvolili ACŽ-AHD (asynchronní cyklické žádosti). 
Tento algoritmus jsme zvolili z důvodu dobré efektivity a jednoduché implementace.
Algoritmus je popsán následovně:

\begin{itemize}
  \item{Každý procesor si udržuje \textbf{lokální čítač} $D$, 
  $0 <= D <= p - 1$, což je index potenciálního dárce.\\
    Počáteční hodnota může být $D = (mytid() + 1)$ mod $p$.}
  \item{Jestliže se procesor stane nečinný, požádá procesor $\#\langle D \rangle$ a inkrementuje cítač $D$.}
  \item{Může se stát, že jeden procesor je žádán více procesory současně, ale pravděpodobně ne mnoha, protože
  požadavky na práci generuje každý procesor nezávisle na ostatních.}
  \item{Z počátku může být komunikace lokální, ale postupně se může stát globální.}
\end{itemize}

Pro implementaci zásobníku jsme oproti sekvenčnímu řešení zvolili spojový seznam, 
protože tato varianta umožňuje odebírání stavů jak ze začátku, tak z konce zásobníku.
Jádro zásobníku tvořené třídou vector jsme zachovali.

Struktura spuštění programu zůstala stejná jako pro sekvenční řešení -- 
jediným povinným argumentem je soubor obsahující vstupní data.
Zbylou část algoritmu jsme založili na sekvenčním řešení, které jsme
pouze rozšířili o meziprocesorovou komunikaci, kterou zajišťuje knihovna MPI.

Knihovna MPI umožnuje přímou komunikaci mezi procesory nezávisle na komunikačních protokolech nižších vrstev.
Taktéž je možné řídit tok programu na základě toho, zda je program spuštěn na 
procesoru mateřském nebo na jiném.
Naše implementace používá formy blokujících funkcí jak pro zasílání, tak pro příjem zpráv z důvodu snazší implementace.

%Popište paralelní algoritmus, opět vyjděte ze zadání a přesně
%vymezte odchylky, zvláště u algoritmu pro vyvažování zátěže, hledání
%dárce, ci ukončení výpočtu.  Popište a vysvětlete strukturu
%celkového paralelního algoritmu na úrovni procesů v MPI a strukturu
%kódu jednotlivých procesů. Např. jak je naimplementována smyčka pro
%činnost procesů v aktivním stavu i v stavu nečinnosti. Jaké jste
%zvolili konstanty a parametry pro škálování algoritmu. Struktura a
%sémantika příkazové řádky pro spouštění programu.

\section{Naměřené výsledky a vyhodnocení}

Pro sekvenční řešení jsme nalezli tři instance problému v zadaném časovém intervalu a 
následně jsme tyto instance zpracovali pomocí paralelního řešení. 

\begin{table}[ht]
  \begin{center}
    \begin{tabular}{| c | c | c |}
      \hline
      $n$ & $k$ & doba běhu úlohy \\ \hline
      350 & 168 & 11 minut, 41 sekund \\ \hline
      320 & 160 & 12 minut, 23 sekund \\ \hline
      155 & 101 & 10 minut, 26 sekund \\
      \hline
    \end{tabular}
    \caption{Tabulka naměřených hodnot pro sekvenční řešení.}
  \end{center}
\end{table}

\begin{table}[ht]
  \begin{center}
    \begin{tabular}{| c | c | c | c | c |}
      \hline
      $n$ & $k$ & doba běhu úlohy     & počet procesorů & zrychlení \\ \hline
      350 & 168 & 5 minut, 43 sekund  & 2               & 2.04      \\ \hline
      350 & 168 & 2 minuty, 52 sekund & 4               & 4.07      \\ \hline
      350 & 168 & 1 minuta, 9 sekund  & 8               & 10.15     \\ \hline
      350 & 168 & 35 sekund           & 16              & 20.02     \\ \hline
      350 & 168 & 24 sekund           & 24              & 29.2      \\ \hline
      350 & 168 & 20 sekund           & 32              & 35.05     \\ \hline
      320 & 160 & 5 minut, 32 sekund  & 2               & 2.23      \\ \hline
      320 & 160 & 2 minuty, 47 sekund & 4               & 4.44      \\ \hline
      320 & 160 & 1 minuta, 8 sekund  & 8               & 10.92     \\ \hline
      320 & 160 & 34 sekund           & 16              & 21.85     \\ \hline
      320 & 160 & 23 sekund           & 24              & 32.3      \\ \hline
      320 & 160 & 19 sekund           & 32              & 39.1      \\ \hline
      155 & 101 & 7 minut, 40 sekund  & 2               & 1.36      \\ \hline
      155 & 101 & 3 minuty, 56 sekund & 4               & 2.65      \\ \hline
      155 & 101 & 1 minuta, 33 sekund & 8               & 6.73      \\ \hline
      155 & 101 & 46 sekund           & 16              & 13.6      \\ \hline
      155 & 101 & 39 sekund           & 24              & 16.05     \\ \hline
      155 & 101 & 26 sekund           & 32              & 24.07     \\
      \hline
    \end{tabular}
    \caption{Tabulka naměřených hodnot pro paralelní řešení.}
  \end{center}
\end{table}

Tyto instance problému byly dále využity pro paralelní řešení a zkoumání zrychlení při paralelním řešení problému.
Z naměřených hodnot je zřejmé, že mezi proměnnýmí $n$ a $k$ existuje určitá závislost. 
Tato závislost se projevila zejména při hledání vhodných instancí problému v daném časovém intervalu, 
přicemž pozměnění jedenoho ze dvou parametrů o několik jednotek při zachování druhého, mělo za následek
několikanásobné prodloužení či zkrácení doby běhu úlohy.

Při hledání vhodných instancí problému jsme často testovali úlohu pro obdobné nebo stejné parametry vstupního grafu,
díky tomu jsme přišli na to, že doba běhu úlohy je závislá na konkrétní podobě grafu.
Tento faktor může hrát taktéž určitou roli v porovnání zrychlení paralelního řešení oproti sekvenčnímu, 
protože v daném časovém intervalu hrála významnou roli -- při časech přes 10 minut byl rozptyl až zhruba 2 minuty.

Většina zrychlení paralelního řešení je způsobena výhodnoným inicialním rozložením práce mezi procesory.
Při implementaci jsme se snažili o celkové rozdělení stavového prostoru mezi všechny procesory, to je pravpodědobně důvodem ideálních výsledků, které přibližně odpovídají lineárnímu zrychlení.
Další možný zdroj zrychlení je přerozdělování práce v průběhu hlavního výpočtu, jelikož některé procesory můžou i přes 
úvodní rovnoměrné rozdělení končit rychleji než ostatní, protože není třeba procházet všechny možné kombinace uzlů,
ale jen ty, které stále splňují podmínku, že jejich uzly tvoří kliku grafu.

Z naměřených výsledků je patrné, že pro paralelní řešení dochází oproti sekvenčnímu
řešení ke zhruba $n$ násobnému zrychlení, kde $n$ je počet procesorů. 
V některých případech je zrychlení dokonce vyšší než lineární, avšak stále zhruba odpovídá lineárnímu zrychlení.
Toto může být podle nás způsobeno konkrétními výhodnými nebo nevýhodnými daty. 
Pro řešení tohoto problému a zvolené instance problému pravděpodobně neexistuje
šance, že by paralelní řešení dosáhlo oproti sekvenčnímu superlineárního zrychlení.

%\begin{table}[ht]
%  \begin{center}
%    \begin{tabular}{| c | c | c | c |}
%      \cline{2-4}
%                 \multicolumn{1}{c|}{} & $n$ & $k$ & doba běhu úlohy \\ \hline
%     \multirow{3}{*}{sekvenční řešení} & 350 & 168 & 11 minut, 41 sekund \\ 
%                                       & 320 & 160 & 12 minut, 23 sekund \\ 
%                                       & 155 & 101 & 10 minut, 26 sekund \\ \hline
%     \multirow{3}{*}{paralelní řešení} & 350 & 168 & 46.7416 sekund pro 12 cpu - jeste upravit\\
%                                       & 320 & 160 & \\
%                                       & 155 & 101 & \\
%      \hline
%    \end{tabular}
%    \caption{Tabulka naměřených hodnot pro sekvenční i paralelní řešení.}
%  \end{center}
%\end{table}


%\begin{enumerate}
%\item Zvolte tři instance problému s takovou velikostí vstupních dat, pro které má
%sekvenční algoritmus časovou složitost kolem 5, 10 a 15 minut. Pro
%meření čas potřebný na čtení dat z disku a uložení na disk
%neuvažujte a zakomentujte ladící tisky, logy, zprávy a výstupy.
%\item Měřte paralelní čas při použití $i=2,\cdot,32$ procesorů na sítích Ethernet a InfiniBand.
%%\item Pri mereni kazde instance problemu na dany pocet procesoru spoctete pro vas algoritmus dynamicke delby prace celkovy pocet odeslanych zadosti o praci, prumer na 1 procesor a jejich uspesnost.
%%\item Mereni pro dany pocet procesoru a instanci problemu provedte 3x a pouzijte prumerne hodnoty.
%\item Z naměřených dat sestavte grafy zrychlení $S(n,p)$. Zjistěte, zda a za jakych podmínek
%došlo k superlineárnímu zrychlení a pokuste se je zdůvodnit.
%\item Vyhodnoďte komunikační složitost dynamického vyvažování zátěže a posuďte
%vhodnost vámi implementovaného algoritmu pro hledání dárce a dělení
%zásobníku pri řešení vašeho problému. Posuďte efektivnost a
%škálovatelnost algoritmu. Popište nedostatky vaší implementace a
%navrhněte zlepšení.
%\item Empiricky stanovte
%granularitu vaší implementace, tj., stupeň paralelismu pro danou
%velikost řešeného problému. Stanovte kritéria pro stanovení mezí, za
%kterými již není učinné rozkládat výpočet na menší procesy, protože
%by komunikační náklady prevážily urychlení paralelním výpočtem.
%
%\end{enumerate}

\section{Závěr}

Během semestru jsme měli možnost vyzkoušet si práci s knihovnou Open MPI
při řešení problému hledání největší kliky v zadaném grafu.
Semestrální práci hodnotíme jako zdařilou, avšak v reálném použití
by pravděpodobně bylo nutné provést další meření pro různě velké instance problémů, 
detailně analyzovat hlavní části programu a případně se pokusit tyto části optimalizovat.

%\section{Literatura}
%
%\appendix

\end{document}
