\documentclass[12pt]{article}
\usepackage{epsf,epic,eepic,eepicemu}
\usepackage[utf8]{inputenc}

\usepackage[czech]{babel}
\usepackage{wrapfig}
\usepackage{graphicx}
\usepackage{epstopdf}

\begin{document}

\begin{center}
  \bf Semestralní projekt MI-PAR 2014/2015:\\[5mm]
      Paralelní algoritmus pro řešení problému\\[5mm]
         Jindřich Štěpánek\\
         Václav Mach\\[2mm]
  magisterské studijum, FIT ČVUT, Kolejní 550/2, 160 00 Praha 6\\[2mm]
  \today
\end{center}

\section{Definice problému a popis sekvenčního algoritmu}

Zadaným problémem bylo nalezení maximální kliky na zadaném grafu.
Pojmem klika grafu je v teorii grafů označen takový maximální podgraf nějakého grafu, 
který je úplným grafem, tzn. jehož všechny vrcholy jsou spojeny hranou se všemi zbylými.
Velikost kliky je dána počtem vrcholů.

\begin{figure}[ht]
  \begin{center}
    \includegraphics[scale=0.5]{clique.pdf}
  \end{center}
  \caption{Největší klika (1,2,5) tohoto grafu je označena červeně.}
\end{figure}

Nalezení maximální kliky grafu tedy znamená, že je nutné postupně procházet celý stavový prostor uzlů grafu
a zkoumat, zda konkrétní množina uzlů tvoří úplný graf. Problém je možné řešit dvěma způsoby, 
buď postupným přidáváním nebo odebíráním uzlů. Rozhodli jsme se pro řešení pomocí přidávání.

Vstupem programu je jednoduchý souvislý neorientovaný neohodnocený graf $G(V, E)$ o n uzlech a m hranách. 
Dále $n$ je přirozené číslo představující počet uzlů grafu $G$, $n >= 5$,
k je přirozené číslo řádu jednotek představující průměrný stupeň uzlu grafu $G$, $n >= k >= 3$ a $r$ kladné reálné číslo takové, že $0 < r < 1$.
Úkolem programu mělo být zjištění, zda graf $G$ obsahuje kliku o velikosti alespoň (rovnou nebo větší) $r * n$ a nalézt největší takovou kliku. 
Výstupem programu měl být seznam uzlů tvořící kliku, popřípadě konstatování, že klika neexistuje. 

Sekvenční algoritmus pro řešení tohoto problému je typu BB-DFS (branch-and-bound Depth-first search) 
s hloubkou stavového stromu omezenou na n. 
Cena řešení, která se maximalizuje, je velikost kliky vzhledem k zadané podmínce. 
Horní mez ceny řešení není známa. Algoritmus skončí, až prohledá celý stavový prostor.
Dolní mez je 2, pokud graf obsahuje aspoň 1 hranu.
Horní mez není známá, ale dá se odhadnout takto: Pokud $G$ obsahuje kliku o velikosti $x$, pak musí obsahovat $x$ vrcholů se stupněm větším nebo rovným $x-1$.

Pro naši implementaci sekvenčního algoritmu jsme zvolili jazyk C++.
Program nejprve přečtě vstupní data, následně inicializuje zásobník realizovaný pomocí třídy vector s datovým typem int.
Dále program už pouze prohledává stavový prostor, dokud má na zásobníku data ke zpracování
a testuje zda konkrétní daná množina tvoří úplný graf. 
Nejvetší nalezenou kliku si uložíme a po skončení prohledávání stavového prostoru ji porovnáme se zadanou podmínkou.



\begin{table}[ht]
  \begin{center}
    \begin{tabular}{| c | c | c |}
      \hline
      $n$ & $k$ & doba běhu úlohy \\ \hline
      350 & 168 & 11 minut, 41 sekund \\ \hline
      320 & 160 & 12 minut, 23 sekund \\ \hline
      155 & 101 & 10 minut, 26 sekund \\
      \hline
    \end{tabular}
    \caption{Tabulka naměřených hodnot pro sekvenční řešení.}
  \end{center}
\end{table}




%Popište problém, který váš program řeší. Jako výchozí použijte text
%zadání, který rozšiřte o přesné vymezení všech odchylek, které jste
%vůči zadání během implementace provedli (např.  úpravy heuristické
%funkce, organizace zásobníku, apod.). Zmiňte i případně i takové
%prvky algoritmu, které v zadání nebyly specifikovány, ale které se
%ukázaly jako důležité.  Dále popište vstupy a výstupy algoritmu
%(formát vstupních a výstupních dat). Uveďte tabulku nameřených časů
%sekvenčního algoritmu pro různě velká data.

\section{Popis paralelního algoritmu a jeho implementace v MPI}

Paralelní algoritmus pro řešení tohoto problému je typu PBB-DFS-V. 
Algoritmus PBB-DFS-V lze popsat následovně: 
Všechny procesory vědí nebo se dozvědí hodnotu horní meze ceny řešení a hodnota dolní meze není známa. 
Pak stačí, aby si každý procesor lokálně udržoval informaci o svém dosud nejlepším řešení. 
Po vyprázdnění všech zásobníků se provede distribuované ukončení výpočtu pomocí algoritmu ADUV a 
pak pomocí paralelní redukce se ze všech nejlepších lokálních řešení vybere globálně nejlepší. 

TODO



\begin{table}[ht]
  \begin{center}
    \begin{tabular}{| c | c | c |}
      \hline
      $n$ & $k$ & doba běhu úlohy \\ \hline
      & & \\ \hline
      & & \\ \hline
      & & \\
      \hline
    \end{tabular}
    \caption{Tabulka naměřených hodnot pro paralelní řešení.}
  \end{center}
\end{table}

%Popište paralelní algoritmus, opět vyjděte ze zadání a přesně
%vymezte odchylky, zvláště u algoritmu pro vyvažování zátěže, hledání
%dárce, ci ukončení výpočtu.  Popište a vysvětlete strukturu
%celkového paralelního algoritmu na úrovni procesů v MPI a strukturu
%kódu jednotlivých procesů. Např. jak je naimplementována smyčka pro
%činnost procesů v aktivním stavu i v stavu nečinnosti. Jaké jste
%zvolili konstanty a parametry pro škálování algoritmu. Struktura a
%sémantika příkazové řádky pro spouštění programu.

\section{Naměřené výsledky a vyhodnocení}

%\subsection{Sekvenční řešení}

Pro sekvenční řešení jsme nalezli tři instance problému v zadaném časovém intervalu a 
následně jsme tyto instance zpracovali pomocí paralelního řešení.

\begin{table}[ht]
  \begin{center}
    \begin{tabular}{| c | c | c |}
      \hline
      $n$ & $k$ & doba běhu úlohy \\ \hline
      350 & 168 & 11 minut, 41 sekund \\ \hline
      320 & 160 & 12 minut, 23 sekund \\ \hline
      155 & 101 & 10 minut, 26 sekund \\ \hline
      & & \\ \hline
      & & \\ \hline
      & & \\
      \hline
    \end{tabular}
    \caption{Tabulka naměřených hodnot pro sekvenční i paralelní řešení.}
  \end{center}
\end{table}

%\begin{enumerate}
%  \item n = 350, k = 168, trvání 11 minut, 41 sekund
%  \item n = 320, k = 160, trvání 12 minut, 23 sekund
%  \item n = 155, k = 101, trvání 10 minut, 26 sekund
%\end{enumerate}

Tyto instance problému byly dále využity pro paralelní řešení a zkoumání zrychlení při paralelním řešení problému.
Z naměřených hodnot je zřejmé, že mezi proměnnýmí $n$ a $k$ existuje určitá závíslost. 
Tato závislost se projevila zejména při hledání vhodných instancí problému v daném časovém intervalu, 
přicemž pozměnění jedenoho ze dvou parametrů o několik jednotek při zachování druhého, mělo za následek
několikanásobné prodloužení či zkrácení doby běhu úlohy.

%Protože hledání vhodných instancí problému bylo časové náročné,a 

%\subsection{Paralelní řešení}
TODO

%\begin{enumerate}
%\item Zvolte tři instance problému s takovou velikostí vstupních dat, pro které má
%sekvenční algoritmus časovou složitost kolem 5, 10 a 15 minut. Pro
%meření čas potřebný na čtení dat z disku a uložení na disk
%neuvažujte a zakomentujte ladící tisky, logy, zprávy a výstupy.
%\item Měřte paralelní čas při použití $i=2,\cdot,32$ procesorů na sítích Ethernet a InfiniBand.
%%\item Pri mereni kazde instance problemu na dany pocet procesoru spoctete pro vas algoritmus dynamicke delby prace celkovy pocet odeslanych zadosti o praci, prumer na 1 procesor a jejich uspesnost.
%%\item Mereni pro dany pocet procesoru a instanci problemu provedte 3x a pouzijte prumerne hodnoty.
%\item Z naměřených dat sestavte grafy zrychlení $S(n,p)$. Zjistěte, zda a za jakych podmínek
%došlo k superlineárnímu zrychlení a pokuste se je zdůvodnit.
%\item Vyhodnoďte komunikační složitost dynamického vyvažování zátěže a posuďte
%vhodnost vámi implementovaného algoritmu pro hledání dárce a dělení
%zásobníku pri řešení vašeho problému. Posuďte efektivnost a
%škálovatelnost algoritmu. Popište nedostatky vaší implementace a
%navrhněte zlepšení.
%\item Empiricky stanovte
%granularitu vaší implementace, tj., stupeň paralelismu pro danou
%velikost řešeného problému. Stanovte kritéria pro stanovení mezí, za
%kterými již není učinné rozkládat výpočet na menší procesy, protože
%by komunikační náklady prevážily urychlení paralelním výpočtem.
%
%\end{enumerate}

\section{Závěr}

%Během semestru jsme m
%
%Celkové zhodnocení semestrální práce a zkušenosti získaných během
%semestru.

TODO

\section{Literatura}

\appendix

\end{document}
